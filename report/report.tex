\documentclass[11pt,letterpaper,titlepage]{article}
\usepackage[margin=2.54cm]{geometry}
\usepackage{textcomp}
\usepackage{hyperref}
\usepackage{listings}
\usepackage{color}
\usepackage{graphicx}



\title{Resume Critic - Documentation}
\author{David Mulatti, Kyle Petrozzi}
\date{}


\begin{document}
    \maketitle


    \section{Purpose}
        \paragraph{}
        The purpose of the website is to be able to upload your
        resume and have it critiqued and be able to critique other resumes with
        comments. Once you read comments, you are able to read them and
        reupload/update your resume. You must create an account to post comments
        and upload your resume in order keep track of each comment and resume.

    \section{Installation}
        \paragraph{}
        To install the website, simply extract the source code into your
        \emph{public\_html} folder. Then, replace the values in
        \emph{dbaccess.php} with the appropriate information for the database
        you'd like to use. Lastly, navigate to ``/admin/create\_db.php'' to
        create the database, then navigate to ``/admin/populate\_db.php'' to
        populate the database with four test users, and an admin account.

        \begin{center}
            \begin{tabular}{ | l | l | }
                \multicolumn{2}{c}{Login Info} \\
                \hline
                User Name & Password \\ \hline
                admin & password \\ \hline
                testone & password \\ \hline
                testtwo & password \\ \hline
                testthree & password \\ \hline
                testfour & password \\
                \hline
            \end{tabular}
        \end{center}

        \paragraph{}
        For a working demo, this site is live at
        \href{334.mulatti.com}{334.mulatti.com}.


    \section{Database Design}
        \paragraph{}
        The database for this site contains three tables: users, comments, and
        ratings. The users table stores user info, including uwinid, hashed
        password, full name, rating on their resume, resume upload date,
        description for their resume, and a boolean indicating whether they have
        uploaded a resume or not.

        \paragraph{}
        The comments table stores an auto incrementing comment id, the uwinid of
        the commenter, the uwinid of the user the comment is for, the date of
        the comment, and the comment itself.

        \paragraph{}
        The ratings table is almost identical to the comments table, but does
        not store the date, and is used to store ratings represented as a double
        rather than comments.

    \section{Site Walkthrough}

        \subsection{Design and CSS}
        \paragraph{}
        Bootstrap is used for the design of this site, with the assistance
        of a kickoff template provided by maxdesign \cite{css}. There are
        modifications made to this, located in /assets/css/styles.css.

        \subsection{Admin Control Panel}
        \paragraph{}
        When logged in as admin, an ``Admin'' option appears in the site header
        that will allow one to access the admin control panel. In this, the
        admin can view all of the tables in the database, edit any user, delete
        any comment, drop the database, create the database and populate the
        database with the test data shown above.

        \paragraph{}
        Also, when logged in as admin, the edit user page is extended to include
        more options, and allows the admin to edit any user they wish.


        \subsection{Header}
        \paragraph{}
        The header is included within every page that is viewed by the user. It
        contains most of the sites navigation features and allows the users to
        easily navigate the website. It is dynamic as it changes with who is
        logged in (such as the admin, admin pages will appear). It contains
        different links as well as the login form. The login button brings you
        to the login.php page. It uses the session variable ``logged\_in'' to
        control most of the dynamic content.

        \subsection{Footer}
        \paragraph{}
        The footer is included in every page, and is simply the closing
        $\langle$body$\rangle$ and $\langle$html$\rangle$ tags, as well as a
        centered message including an emoji. The emoji is generated using
        Twitter's \emph{twemoji} plugin \cite{twemoji}.

        \subsection{Create Account}
        \paragraph{}
        Here, there is a is a web form that takes a uwinid, password, and full
        name to create a new account. There is also a captcha, provided using
        Google's reCAPTCHA api, to verify that the person signing up for an
        account is indeed human. The form uses the jQuery Validation plugin
        \cite{validate} to validate all the fields, including AJAX calls to the
        database to ensure that a username is not taken. It will also ensure
        that the password is at least 8 characters long, and that the full name
        only has alphanumeric characters in it.

        \paragraph{}
        On submission, all fields are sent as POST variables to
        \emph{createaccount\_go.php} which will insert a new user into the
        database. The password is hashed using the \emph{password\_hash()}
        function, provided by ircmaxwell \cite{password}.

        \subsection{Login and Logout}
        \paragraph{}
        The login.php is relatively simple. It compares the password sent within
        the form to the hashed password stored in the database associated with
        the name sent in the form. If it matches it sets the \emph{logged\_in}
        session variable to 1, and then redirects back to the index page. It
        also checks if the user logging in is the admin, and if so, changes the
        session variable to 3. It also has an error page if the wrong username
        or password is entered.

        \subsection{Edit User}
        \paragraph{}
        The edit user link appears in the header whenever a user is logged in. A
        user can then edit their Full Name or their password. If an admin is
        logged in, then the edit user page is extended, allowing the admin to
        change many more options for any user they wish.

        \subsection{Resume Upload}
        \paragraph{}
        This page is used to upload/edit your own resume page. If you are logged
        in you can add a resume and description. If you already have a resume
        uploaded, it shows it to you, and auto fills the description with your
        last description used. When you use the upload button it calls the
        upload.php class. This class checks if the file uploaded is a
        application/pdf MIME type, if it comes from the POST function, and
        finally if it is not too large of a file. If all of these checks pass,
        then it saves the file as \emph{uwinid}.pdf, \emph{uwinid} being the
        username of the current user. It also updates the upload date,
        \emph{has\_uploaded}, and the description columns within the database to
        the user who uploaded.
        It also will add a comment to the resume, notifying that a new resume
        has been uploaded.
        It then redirects back to the resume viewer page.
        It also has error pages that tell the user what went wrong if there is a
        upload error.

        \subsection{Resume Viewer}
        \paragraph{}
        When a name on the homepage is clicked, the user is brought to a page
        which displays the resume as an embedded object on the left, and the
        description, upload date, and comments on the resume on the right. Above
        the resume there is a current star rating of the resume, gathered by
        what the users have voted.

        \paragraph{}
        If the user is logged in, below the comments there is a textarea to add
        their own comment, and an option to rate the resume out of five stars.
        The star rating system is powered by a jQuery plugin for bootstrap made
        by GitHub user \emph{kartik-v} \cite{stars}.

        \subsection{Contact Us}
        \paragraph{}
        The ``Contact Us'' page displays our names, some information about
        ourselves, and our email addresses. There is also a link to the GitHub
        page that our code is available on, and a form including a TinyMCE
        \cite{tinymce} editor to send an email directly to us using the PHP
        \emph{mail()} function.

        \paragraph{}
        On submission of the contact form, the user is redirected to
        \emph{contact\_go.php} which takes the POST variables from the form,
        validates the captcha, and processes the message given from the TinyMCE
        editor through XSLT to strip out everything that isn't a
        $\langle$b$\rangle$, $\langle$u$\rangle$, or $\langle$i$\rangle$
        tag.






    \newpage
    \begin{thebibliography}{9}

        \bibitem{css}
          russmaxdesign,
          \emph{Bootstrap Kickoff Template}\\
          Available at:
          \href{https://github.com/russmaxdesign/01-bootstrap-kickoff-template}
               {https://github.com/russmaxdesign/01-bootstrap-kickoff-template}

        \bibitem{twemoji}
        Twitter,
        \emph{Twitter Emoji For Everyone}\\
        Available at: \href{https://github.com/twitter/twemoji}
                        {https://github.com/twitter/twemoji}

        \bibitem{validate}
        jQuery Validation\\
        Available at: \href{http://jqueryvalidation.org/}
                           {http://jqueryvalidation.org/}

        \bibitem{password}
        ircmaxwell,
        \emph{password\_compat}\\
        Available at: \href{https://github.com/ircmaxell/password_compat}
                           {https://github.com/ircmaxell/password\_compat}

       \bibitem{stars}
       kartik-v,
       \emph{bootstrap-star-rating}\\
       Available at: \href{https://github.com/kartik-v/bootstrap-star-rating}
                          {https://github.com/kartik-v/bootstrap-star-rating}

       \bibitem{tinymce}
       TinyMCE\\
       Available at: \href{https://www.tinymce.com/}
                          {https://www.tinymce.com/}

    \end{thebibliography}



\end{document}
