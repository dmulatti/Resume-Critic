\documentclass[11pt,letterpaper,titlepage]{article}
\usepackage[margin=2.54cm]{geometry}
\usepackage{textcomp}
\usepackage{hyperref}
\usepackage{listings}
\usepackage{color}
\usepackage{graphicx}



\title{Resume Critic - Documentation}
\author{David Mulatti, Kyle Petrozzi}
\date{}


\begin{document}


    \maketitle
    \section{Purpose}
    \paragraph{}
    The purpose of the website is to be able to upload your resume and have it
    critiqued and be able to critique other resumes with comments. Once you read
    comments, you are able to read them and reupload/update your resume. You
    must create an account to post comments and upload your resume in order keep
    track of each comment and resume.

    \section{Installation}
    \paragraph{}
    To install the website, simply extract the source code into your
    \emph{public\_html} folder. Then, replace the values in \emph{dbaccess.php}
    with the appropriate information for the database you'd like to use.
    Lastly, navigate to ``/admin/create\_db.php'' to create the database, then
    navigate to ``/admin/populate\_db.php'' to populate the database with four
    test users, and an admin account.

    \begin{center}
        \begin{tabular}{ | l | l | }
            \multicolumn{2}{c}{Login Info} \\
            \hline
            User Name & Password \\ \hline
            admin & password \\ \hline
            testone & password \\ \hline
            testtwo & password \\ \hline
            testthree & password \\ \hline
            testfour & password \\
            \hline
        \end{tabular}
    \end{center}

    \section{Site Walkthrough}

    \subsection{Admin Control Panel}
    \paragraph{}
    When logged in as admin, an ``Admin'' option appears in the site header
    that will allow one to access the admin control panel. In this, the admin can
    view all of the tables in the database, edit any user, delete any comment,
    drop the database, create the database and populate the database with the
    test data shown above.

    \paragraph{}
    Also, when logged in as admin, the edit user page is extended to include
    more options, and allows the admin to edit any user they wish.


    \subsection{header.php}
    \paragraph{}
    The header is included within every page that is viewed by the user. It
    contains most of the sites navigation features and allows the users to
    easily navigate the website. It is dynamic as it changes with who is logged
    in (such as the admin, admin pages will appear). It contains different
    links as well as the login form. The login button brings you to the
    login.php page. It uses the session variable ``logged\_in'' to control most
    of the dynamic content.

    \subsection{login.php and logout.php}
    \paragraph{}
    The login.php is relatively simple. It compares the password sent within the
    form to the hashed password stored in the database associated with the name
    sent in the form. If it matches it sets the ``logged\_in'' session variable
    to 1, and then redirects back to the index page. It also checks if the user
    logging in is the admin, and if so, changes the session variable to 3. This
    uses SQL and stores who is logged in. It also has an error page if the wrong
    username or password is entered.

    \subsection{resumeupload.php}
    \paragraph{}
    This page is used to upload/edit your own resume page. If you are logged in
    you can add a resume and description. If you already have a resume uploaded,
    it shows it to you, and auto fills the description with your last
    description used. When you use the upload button it calls the upload.php
    class. This class checks if the file uploaded is a application/pdf MIME
    type, if it comes from the POST function, and finally if it is not too large
    of a file. If all of these checks pass, then it saves the file as
    \emph{uwinid}.pdf, \emph{uwinid} being the username of the current user. It
    also updates the upload date, \emph{has\_uploaded}, and the description
    columns within the database to the user who uploaded. It then redirects back
    to the resume viewer page. It also has error pages that tell the user what
    went wrong if there is a upload error.




\end{document}
